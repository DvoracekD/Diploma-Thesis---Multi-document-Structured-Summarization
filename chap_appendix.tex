\label[attached-files]
\app Attachments

\sec Code Base \& Data

The attachments contain Python notebooks and scripts that were used for the practical part of this thesis. It is a standard Python project that can be installed using: "pip install -e ."

\begitems
* "multi_sum/": Main Python package containing modules.
  \begitems \style x
  * "comparison/": Notebooks for comparing different model versions and temperature settings.
  * "data_analysis/": Notebooks and scripts for data exploration, deduplication, headline analysis, and temporal/topic analysis.
  * "eval/": Evaluation scripts and notebooks for assessing summarization quality.
  * "extraction_pipeline/": Extraction pipeline code and development notebooks.
  * "utils/": Utility scripts for data processing, stemming, MLflow integration, and matching.
  \enditems

* "evaluation_dataset/"  
  Contains datasets for evaluation, organized by topic 
  
  ("cesky_lev", "covid", "peking", "summit", "volby", "oteplovani"). Each topic includes:
  \begitems  \style x
  * "<topic>_articles.json": Source articles.
  * "<topic>_gt.json": Ground truth summaries.
  * "articles_gt/": Additional ground truth data for single topics.
  \enditems

* "visualization/"  
  Jupyter notebooks for visualizing model results, metrics, and experiments.

* "prompts/"  
  Prompt templates for different model versions ("v1", "v2", "v3"), organized by entity type (events, locations, organizations, people).

* "schemas/"  
  JSON schema files defining the structure of extracted entities.

* "pyproject.toml"  
  Project dependencies and configuration.

* "README.md"  
  Instructions and overview of the repository.
\enditems

\sec Experiment logs

Folder "mlruns" with records of all experiments compatible with MLflow is provided. The experiments contain both input and output, as well as parameters, prompts, API logs including thinking outputs, the resulting predictions and monitored metrics.

\sec Thesis source code
The source code of the diploma thesis document is also included. The compilation instructions can be found on the \url{https://petr.olsak.net/ctustyle-e.html} website.
